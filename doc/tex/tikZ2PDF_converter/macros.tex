\usepackage{amssymb}			%% The amssymb package provides various useful mathematical symbols
\usepackage{amsmath, xparse}
\usepackage{varwidth}
\usepackage{lineno}
\usepackage{relsize}
\usepackage{scalerel}
\usepackage{graphicx}
\usepackage{todonotes}
\usepackage{float}
\usepackage{tikz}
\usetikzlibrary{arrows,shapes,trees}
\usepackage{color}
\usepackage{booktabs}
\usepackage{multirow}
\usepackage{multicol}
\usepackage{siunitx}
\usepackage{pgfplots}
\usepackage{array}
\usepgfplotslibrary{groupplots}
\pgfplotsset{compat=1.5}
\usepackage{subcaption}
\usepackage[export]{adjustbox}
\usepackage{mwe}
% \usepackage{graphbox} %loads graphicx package
% \usepackage[caption=false]{subfig}	% this is incompatible with subcaption **** Don't use
\usepackage{lipsum} % just for the example
%\usepackage{cleveref}
\usepackage{url}
\usepackage{bm}
\usepackage{pbox}
\usepackage{longtable}
\usetikzlibrary{calc}
\usetikzlibrary{patterns}
\usetikzlibrary{decorations.pathmorphing}
%%SUKU \usetikzlibrary{quotes,angles}
\usepackage{tkz-euclide}
%%SUKU \pgfplotsset{compat=1.12}                                
\usetkzobj{all}
\usepackage{tikz-3dplot}
\usepackage[ocgcolorlinks=false]{hyperref}
\hypersetup{   
    urlcolor=red,
}
\usepackage{algorithmicx}
\usepackage[ruled]{algorithm}
\usepackage{algpseudocode}
\usepackage{xspace}
\algnewcommand\OR{\textbf{or}\xspace}
\algnewcommand\AND{\textbf{and}\xspace}
\algnewcommand\NOT{\textbf{not}\xspace}
\algnewcommand\Break{\textbf{break}\xspace}
\algnewcommand\Continue{\textbf{continue}\xspace}

%% The lineno packages adds line numbers. Start line numbering with
%% \begin{linenumbers}, end it with \end{linenumbers}. Or switch it on
%% for the whole article with \linenumbers.
%% \usepackage{lineno}

%
\pgfplotsset{compat=1.7}
\tikzset{>=latex}
\renewcommand{\Re}{{\rm{I\!R}}}
\renewcommand{\arraystretch}{1.4}
\renewcommand{\i}{i}
\newcommand{\eref}[1]{(\ref{#1})}
\newcommand{\fref}[1]{Fig.~\ref{#1}}
\newcommand{\tref}[1]{Table~\ref{#1}}
\newcommand{\sref}[1]{Section~\ref{#1}}
\newcommand{\vm}[1]{\bm{#1}}
\newcommand{\vx}{\vm{x}}
\newcommand{\bsym}[1]{\bm{#1}}
\newcommand{\blambda}{\bsym{\lambda}}
\newcommand{\abs}[1]{{\lvert#1\rvert}}
\newcommand{\norm}[1]{{\lVert#1\rVert}}
\newcommand{\suku}[1]{\Red{#1}}
\newcommand\floor[1]{\lfloor#1\rfloor}
\newcolumntype{P}[1]{>{\centering\arraybackslash}p{#1}}
%\newcommand{\crefrangeconjunction}{--}
\DeclareMathOperator*{\A}{ \mathlarger{\mathlarger{\mathlarger{\vm{\mathsf{A}}}}} }
\def\bibsection{\section*{References}}
%
%%%%%% end: caption/subcaption setup %%%%%%  
\tikzset{
    rotate around with nodes/.style args={#1:#2}{
        rotate around={#1:#2},
        set node rotation={#1},
    },
    rotate with/.style={rotate=\qrrNodeRotation},
    set node rotation/.store in=\qrrNodeRotation,
}
%%% START MACRO FOR ANNOTATION OF TRIANGLE WITH SLOPE %%%.
\newcommand{\logLogSlopeTriangle}[5]
{
    % #1. Relative offset in x direction.
    % #2. Width in x direction, so xA-xB.
    % #3. Relative offset in y direction.
    % #4. Slope d(y)/d(log10(x)).
    % #5. Plot options.

    \pgfplotsextra
    {
        \pgfkeysgetvalue{/pgfplots/xmin}{\xmin}
        \pgfkeysgetvalue{/pgfplots/xmax}{\xmax}
        \pgfkeysgetvalue{/pgfplots/ymin}{\ymin}
        \pgfkeysgetvalue{/pgfplots/ymax}{\ymax}

        % Calculate auxilliary quantities, in relative sense.
        \pgfmathsetmacro{\xArel}{#1}
        \pgfmathsetmacro{\yArel}{#3}
        \pgfmathsetmacro{\xBrel}{#1-#2}
        \pgfmathsetmacro{\yBrel}{\yArel}
        \pgfmathsetmacro{\xCrel}{\xArel}
        %\pgfmathsetmacro{\yCrel}{ln(\yC/exp(\ymin))/ln(exp(\ymax)/exp(\ymin))} % REPLACE THIS EXPRESSION WITH AN EXPRESSION INDEPENDENT OF \yC TO PREVENT THE 'DIMENSION TOO LARGE' ERROR.

        \pgfmathsetmacro{\lnxB}{\xmin*(1-(#1-#2))+\xmax*(#1-#2)} % in [xmin,xmax].
        \pgfmathsetmacro{\lnxA}{\xmin*(1-#1)+\xmax*#1} % in [xmin,xmax].
        \pgfmathsetmacro{\lnyA}{\ymin*(1-#3)+\ymax*#3} % in [ymin,ymax].
        \pgfmathsetmacro{\lnyC}{\lnyA+#4*(\lnxA-\lnxB)}
        \pgfmathsetmacro{\yCrel}{\lnyC-\ymin)/(\ymax-\ymin)} % THE IMPROVED EXPRESSION WITHOUT 'DIMENSION TOO LARGE' ERROR.

        % Define coordinates for \draw. MIND THE 'rel axis cs' as opposed to the 'axis cs'.
        \coordinate (A) at (rel axis cs:\xArel,\yArel);
        \coordinate (B) at (rel axis cs:\xBrel,\yBrel);
        \coordinate (C) at (rel axis cs:\xCrel,\yCrel);

        % Draw slope triangle.
        \draw[#5]   (A)-- node[pos=0.01,anchor=south east] {1}
                    (B)-- 
                    (C)-- node[pos=0.5,anchor=west] {#4}
                    cycle;
    }
}
\tikzset{
    arrowMe/.style={
        postaction=decorate,
        decoration={
            markings,
            mark=at position .5 with {\arrow[thick]{#1}}
        }
    }
}
\pgfdeclarepatternformonly[\LineSpace]{my north east lines}{\pgfqpoint{-1pt}{-1pt}}{\pgfqpoint{\LineSpace}{\LineSpace}}{\pgfqpoint{\LineSpace}{\LineSpace}}%
{
    \pgfsetlinewidth{0.4pt}
    \pgfpathmoveto{\pgfqpoint{0pt}{0pt}}
    \pgfpathlineto{\pgfqpoint{\LineSpace + 0.1pt}{\LineSpace + 0.1pt}}
    \pgfusepath{stroke}
}

\newdimen\LineSpace
\tikzset{
    line space/.code={\LineSpace=#1},
    line space=3pt
}

\pgfplotsset{
    node near coord/.style args={#1/#2/#3}{% Style for activating the label for a single coordinate
        nodes near coords*={
            \ifnum\coordindex=#1 #2\fi
        },
        scatter/@pre marker code/.append code={
            \ifnum\coordindex=#1 \pgfplotsset{every node near coord/.append style=#3}\fi
        }
    },
    nodes near some coords/.style={ % Style for activating the label for a list of coordinates
        scatter/@pre marker code/.code={},% Reset the default scatter style, so we don't get coloured markers
        scatter/@post marker code/.code={},% 
        node near coord/.list={#1} % Run "node near coord" once for every element in the list
    }
}




%%% END MACRO FOR ANNOTATION OF TRIANGLE WITH SLOPE %%%.
